\documentclass{article}
\usepackage{amssymb, amsmath, amsthm}
\usepackage[margin=1in]{geometry}
\usepackage{verbatim}
\usepackage{graphicx}
\usepackage{hyperref} % \url \href

\newtheorem{definition}{Definition}
\newtheorem{theorem}{Theorem}

\begin{document}

\title{Molecular orbitals and symmetrization}
\author{Wenhao Zhang}
\date{\today}
\maketitle

\section{Introduction}

It is inefficient to learn rules using try and error, while the rules can be 
simply stated. 
Similarly, if we want to achieve the best results using machine learning, which 
basically learn from statistic distribution of data, 
it is important that we include as much concrete rules as possible. 
This should greatly improve the accuracy of ML prediction while simplify the 
problems for ML so that fewer data are necessary.
To let machines predict the electronic structure of moleculars, for example, the 
energy levels of the electronic states, 
one way is to input the molecular structure to ML and ask it to output the energies. 
However, this naive approach suffer from the problems that 1) this 'ab-initio' approach
is too complicated to be formulated as a single function that we can hopefully understand 
and 2) since we ourselves cannot understand it, it is impossible for us to inspect the 
machine learning model: is machine really learning anything important, or it just 
found some other patterns that fits the data but does not agree with our physical 
or chemical understanding. 

Using results from group theory, it is possible to simplify the Hamiltonians of 
a molecular systems. Such approach utilize the symmetry of moleculars and provide
rules that quantum mechanical interactions have to obey. 
For example, a 's' type wavefunction can not interact with the 'p' type wavefunction,
because their intergrals are strictly zero. 
Furthermore, using group theory analysis using only symmetry information, we can find 
our the degeneracy that are exact. 
These informations reduce the size of the problem considerably. As a consequence, to 
find the energy of the states, we now only need to consider the remaining interactions 
that are allowed with the know form of the states, and the problem is then largely 
an electro-static one, which should be simple enough for machine learning to solve.
We try to formulate the ideas in the following sections

\section{Groups, representations and basis}
The symmetry operation that map a molecular onto itself form a group $G$. From a group,
we define a representation of group $G$ as a mapping:
\begin{definition}
    A linear \emph{representation} of a group $G$ on a vector space $V$ is a homomorphism:
    \[\rho\colon G \to \text{GL}(V)\]
    where homomorphism is a mapping that satisfies $\rho(g_1g_2) = \rho(g_1)\rho(g_2)$.
\end{definition}
Representations are not unique: a change of basis leads to an equivalent representation 
and the direct sum of two representation is again a representation. However, representations
can be uniquely reduced to so called \emph{irreducible representations}.
We define the notion of \emph{invariant vector space}:
\begin{definition}
    For any vector $v$ in $V$, for any $g\in G$, if we have $\rho(g)v \in V$, then $V$
    is an invariant vector space.
\end{definition} 
For a representation with vector space $V$, $V$ is an invariant to the group $G$. 
If there exist a vector space $W\subset V$ which is also invariant to $G$, then we call $W$
an \emph{invariant subspace}. 
Furthermore, there exist another invariant subspace $U$ of $V$ so that $V$ is given by a direct sum:
\begin{equation}
    V = U \oplus W
\end{equation}
By splitting the vector space $V$ into invariant subspaces until it is not possible, we find the 
\emph{irreducible representation} by the reducible representation on $V$, and the matrix form 
of the reducible representation can be written as the block diagonal form
\begin{equation}
    \rho^{\Gamma}(g) = \left(  
        \begin{matrix}
            \rho^{\Gamma_1} (g) & 0 & \cdots\\
            0 & \rho^{\Gamma_2} (g) & \\
            \vdots &  & \ddots
        \end{matrix}
    \right)
\end{equation}
where $\Gamma$ denote the reducible representation and $\Gamma_i\ (i = 1, \dots)$ index 
the irreducible representations.

\begin{thebibliography}{99}
    \bibitem{} 
\end{thebibliography}

\appendix 
% on the same page, section is numberd with A, B automatically
\section{}


\end{document}
