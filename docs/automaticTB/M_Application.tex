\documentclass{article}
\usepackage{amssymb, amsmath, amsthm}
\usepackage[margin=1in]{geometry}
\usepackage{verbatim}
\usepackage{graphicx}
\usepackage{hyperref} % \url \href
\usepackage{docmute}
\usepackage{blkarray}

\newtheorem{definition}{Definition}
\newtheorem{theorem}{Theorem}
\DeclareMathOperator{\spn}{Span}

\usepackage[style=chem-acs ,backend=bibtex, sorting=none]{biblatex}
\addbibresource{autoTB.bib}


\begin{document}

\section{Application}
\subsection{Tight-binding band structure in Pervoskite}

We describe the parameters in the cubic Halide Perovskite described in 
the work of Boyer et al.\cite{boyer-richard_symmetry-based_2016} as the 
our reference model. The parameters are as follows:
\begin{table}[h]
    \centering
    \caption{Tight binding parameters, Halide Perovskite Semiconductors}
    \label{T:TBP}
    \begin{tabular}{|c|c|c|}
        \hline
        Atom & Positions(crystal coord.) & Basis functions \\
        \hline
        B & $(0.0,0.0,0.0)$ & $|S_0\rangle$, $|X_0\rangle$, $|Y_0\rangle$, $|Z_0\rangle$ \\
        X & $(0.5,0.0,0.0)$ & $|S_1\rangle$, $|X_1\rangle$, $|Y_1\rangle$, $|Z_1\rangle$ \\
        X & $(0.0,0.5,0.0)$ & $|S_2\rangle$, $|X_2\rangle$, $|Y_2\rangle$, $|Z_2\rangle$ \\
        X & $(0.0,0.0,0.5)$ & $|S_3\rangle$, $|X_3\rangle$, $|Y_3\rangle$, $|Z_3\rangle$ \\
        \hline
    \end{tabular}
\end{table}
We have in total 16 atomic orbitals in the unit cell. In a $3\times 3\times 3$ supercell
which can include all the nearest neighbor interaction, We would have in total $N_s \times 16 \times 16$ parameters. 
Restricting the interaction to first order, only 304 interactions left. 
Now, we use the fact that orbitals belong to different symmetry cannot form bonding, 
so that matrix elements like $\langle S_0 | H | P_0 \rangle$ is zero, we can reduce the 
number of non-zero parameters. 
Finally, we use symmetry relationship to obtain relationship in bonding parameters. 
The interaction can be parameterized using only 9 parameters:
\begin{align*}
    E_{s0} &= \langle S_0 | H | S_0 \rangle \\
    E_{s1} &= \langle S_1 | H | S_1 \rangle \\
    E_{p0} &= \langle X_0 | H | X_0 \rangle \\
    E_{p1} &= \langle X_1 | H | X_1 \rangle 
\end{align*}
and 
\begin{align*}
    V_{ss} &= \langle S_0 | H | S_1 \rangle  \\
    V_{s0p1} &= \langle S_0 | H | X_1 \rangle \\
    V_{p0s1} &= \langle X_0 | H | S_1 \rangle \\
    V_{pp\sigma} &= \langle X_0 | H | X_1 \rangle \\
    V_{pp\pi} &= \langle Y_0 | H | Y_1 \rangle \\
\end{align*}

\end{document}