\documentclass{article}
\usepackage{amssymb, amsmath, amsthm}
\usepackage[margin=1in]{geometry}
\usepackage{verbatim}
\usepackage{graphicx}
\usepackage{hyperref} % \url \href
\usepackage{docmute}

\newtheorem{definition}{Definition}
\newtheorem{theorem}{Theorem}
\DeclareMathOperator{\spn}{Span}

\usepackage[style=chem-acs ,backend=bibtex, sorting=none]{biblatex}
\addbibresource{autoTB.bib}

\begin{document}

\section{INTRODUCTION}

Symmetry plays an important role in chemistry, offering great simplifications in many 
cases, compared to the blind, mechanical solution\cite{albright_orbital_2013}. 
It provide strict rules that both physical properties and quantum mechanical interactions
follow. For example, electrical conductivity tensor are diagonal with equivalent values for 
compounds with cubic structure because of the rotation symmetry between $a$, $b$ and $c$
lattice vector. In microscopic case, energies of wavefunctions will be degenerate if 
they are related by symmetry. Recently, symmetry properties are also gaining importance in 
the context of machine learning (ML) in the field of image recognization and chemistry. 
Many symmetry--aware ML methods are purposed and some of the notable examples are 
Steerable CNNs and group equivariant networks by Cohen and Welling, 3D Steerable CNNs 
by Weiler and DTNN by Thomas. However, we note that these methods in general explore rotational 
equivariance provided by the mathematic framework of convolution operation and 
spherical harmonics, but they do not utilize symmetry classification provided by 
representation theory and characters. 

In this work, we explore the use of group representations in the context of machine learning 
for chemistry. Machine learning has recently became an important methods in computational chemistry which 
aim to provide a mapping between structures and properties. However, the pitfall in many of the 
methods that predict physical properties is the difficult to interpreted the results since physical 
properties are often distant derivative of electronic properties, which originate from crystal structure 
in a complex way. Therefore, it is important that we should not ignore those intermediate phases in the 
structure properties relationship. On the other hand, prediction of electronic structure in solids 
are still a challenge, whose information should be enough for the derivation of properties. 
Here, instead of directly attack the ML problem, we purpose the method to simplify the burden of 
machine learning by providing a minimum set of machine learnable parameters. Once the suitable values 
of these parameters are known, a tightbinding model can be constructed and electronic structure can be solved.
For example, band dispersion, density of states and transport properties.

Our approach is based on the symmetry adapted linear combination (SALCs) of atomic 
orbitals and the symmetry restrict between their interactions. As will be detailed 
later, interaction between symmetry adapted molecular functions is strictly zero if
they have different symmetry properties. This allow a more than one order of magnititude reduction of 
number of free parameters for the tightbinding interactions. Furthermore, using this 
approach, the obtained Hamiltonian is automatically symmetrized so that the derived 
physical properties will also be symmetrized. 

The values that require machine learning prediction are orbital interactions between two 
atomic orbitals $H_{\mu\nu}$ where the $\mu$ and $\nu$ index the orbitals. 
In the simpliest isolated case, interactions can be given by the H\"uckel fromula as 
$H_{\mu\nu}\propto S_{\mu\nu}(H_{\mu\mu}+H_{\nu\nu})$ i.e., depending on the overlap intergrals
which follows from the geometry and nature between the two orbitals. In solids, they further 
depend on the presence of neighboring atomic environment. The machine learning prediction would 
take the local structure information and orbital information as inputs to map values for 
the interaction, which should be in a rotational invariant way. Here, we note that the 
method purposed in this work provide a framework for a \emph{model}, whose aim is to provide 
reasonably accurate electronic structure. The values such as interaction energies $H_{\mu\nu}$
should be considered as parameters, instead of physical values.


\end{document}
