\documentclass{article}
\usepackage{amssymb, amsmath, amsthm}
\usepackage[margin=1in]{geometry}
\usepackage{verbatim}
\usepackage{graphicx}
\usepackage{hyperref} % \url \href
\usepackage{docmute}

\newtheorem{definition}{Definition}
\newtheorem{theorem}{Theorem}
\DeclareMathOperator{\spn}{Span}

\usepackage[style=chem-acs ,backend=bibtex, sorting=none]{biblatex}
\addbibresource{autoTB.bib}

\begin{document}

\section{INTRODUCTION}

It is inefficient to learn rules using try and error, while the rules can be 
simply stated. 
Similarly, if we want to achieve the best results using machine learning, which 
basically learn from statistic distribution of data, 
it is important that we include as much concrete rules as possible,
which should greatly improve the accuracy of ML prediction while simplify the 
problems for ML so that fewer data are necessary.

To let machines predict the energy levels of the electronic states of moleculars, for example,
one way is to input the molecular structure to ML and ask it to output the energies. 
However, this naive approach suffer from the problems that 1) this `ab-initio' approach
is too complicated to be formulated as a simple function that we can understand 
and 2) if we ourselves cannot understand it, it is more difficult for us to inspect the 
machine learning output: Is machine really learning anything important, or it just 
found some other patterns that fits the data but does not agree with our physical 
or chemical understanding?

Using results from group theory, it is possible to simplify the Hamiltonians of 
a molecular systems. Such approach utilize symmetry and provide
rules that quantum mechanical interactions have to obey. 
In the simplest case, a $s$ type wavefunction can not interact with a $p$ type wavefunction on the 
same atom, because their intergrals are strictly zero. 
Furthermore, group theory analysis force the degeneracy of the energy levels.
These informations reduce the size of the problem considerably. As a consequence, to 
find the energy of the states, we now only need to consider the remaining interactions 
that are allowed between MOs, and the problem is then largely 
an electro-static one, which should be simple enough for machine learning to solve.
We try to formulate the ideas in the following sections


\end{document}
