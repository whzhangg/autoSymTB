\documentclass{article}
\usepackage{amssymb, amsmath, amsthm}
\usepackage[margin=1in]{geometry}
\usepackage{verbatim}
\usepackage{graphicx}
\usepackage{hyperref} % \url \href
\usepackage{docmute}

\newtheorem{definition}{Definition}
\newtheorem{theorem}{Theorem}
\DeclareMathOperator{\spn}{Span}

\usepackage[style=chem-acs ,backend=bibtex, sorting=none]{biblatex}
\addbibresource{autoTB.bib}

\begin{document}

\section{Machine Learning}

\subsection{Complex coefficients}
Certain groups have complex characters. If two MOs correspond to such a pair, it is always possible to 
find a linear combination of them so that the resulting MOs are real functions. 

\subsection{Inputs and Criteria}
In this section, we describe the inputs for the machine learning methods as well as the 
requirement for machine learning models. 
The input of machine learning is a pair of MOs, one located on the atoms in the origin
and one formed from a set of symmetry equivalent orbitals on the neighboring atoms. We note that 
for the Hamiltonian energies $H_{ij} = H_{ji}$ if all matrix elements are real so that 
in principle the machine learning model need to be invariant to a permutation between the 
pair of input MOs. However, in practice, the bar $\langle \psi |$ always correspond to MOs 
from that origin atom and $|\psi\langle$ either correspond to 





\end{document}
