\documentclass{article}
\usepackage{amssymb, amsmath, amsthm}
\usepackage[margin=1in]{geometry}
\usepackage{verbatim}
\usepackage{graphicx}
\usepackage{hyperref} % \url \href
\usepackage{docmute}

\newtheorem{definition}{Definition}
\newtheorem{theorem}{Theorem}
\DeclareMathOperator{\spn}{Span}

\usepackage[style=chem-acs ,backend=bibtex, sorting=none]{biblatex}
\addbibresource{autoTB.bib}

\begin{document}

\section{Finding the representations and Symmetry adopted basis functions}
\subsection{Molecular orbitals}
Using character table and the orthogonality theorem for characters, we can find the basis functions 
of the Hamiltonians knowing only the symmetry property of the structure. 

We start by finding the vector space of all the participating atomic orbitals and their transformation 
properties under group operation, then we decompose the reducible representation to irreducible ones. 
We follow the example of Dresselhaus\cite{dresselhaus_group_2008} and consider the group $D_{3h}$ with 
the following character table \ref{T:ct}
\begin{table}[h!]
    \centering
    \caption{Character table of $D_{3h}$}
    \begin{tabular}{|c|c|c|c|c|c|c|}
        \hline
                & $E$ & $\sigma_h$ & 2$C_3$ & 2$S_3$ & 2$C_2'$ & 3$\sigma_v$ \\ \hline
         $A_1'$ &  1  &  1         &  1     &  1     &   1     &   1         \\
         $A_2'$ &  1  &  1         &  1     &  1     &  -1     &  -1         \\
         $A_1''$&  1  & -1         &  1     & -1     &   1     &  -1         \\
         $A_2''$&  1  & -1         &  1     & -1     &  -1     &   1         \\
         $E'$   &  2  &  2         &  -1    & -1     &   0     &   0         \\
         $E''$  &  2  & -2         &  -1    &  1     &   0     &   0         \\ \hline
    \end{tabular}
    \label{T:ct}
\end{table}

Suppose we have four atoms, one on the rotation axis and three other that transform to each other 
by three fold rotation, and $s$ and $p$ orbitals on each of the atoms. Indexing each atomic 
centered functions using atomic index $i$ and orbital index $l,m$ as $| i,l,m \rangle$ 
(or $| i,s \rangle, \dots, |i,p_x\rangle$ if convenient), we form 
a vector space with dimension $4\times4 = 16$:
\begin{equation}
    V = \spn(| 1,s \rangle, | 1,p_x \rangle ,\dots, | 4,p_y \rangle, | 4,p_z \rangle)
\end{equation}
We can find the matrix elements of the symmetry group in this space using the following equation:
\begin{equation}
    \rho(g)_{ij} = \langle i | P_g | j \rangle
\end{equation}
where $\rho(g)$ is the matrix in the vector space of atomic basis function and $P_g$ is the 
operators that transform the basis functions. 
After generating the matrix of each symmetry operation, we find the character of the representation
in this atomic orbital basis:
\begin{gather*}
    \chi^{AOs}(E) = 16,\ \chi^{AOs}(\sigma_h) = 8,\ \chi^{AOs}(C_3) = 1\\ 
    \chi^{AOs}(S_3) = -1,\ \chi^{AOs}(C_2') = 0,\ \chi^{AOs}(\sigma_v) = 4
\end{gather*}
The uniqe decomposition of the reducible representation into the irreducible ones are given by:
\begin{gather}
    a_i = \frac{1}{h} \sum_{k} N_k \chi^{\Gamma_i}(C_k)^* \chi(C_k)
\end{gather}
Decomposing the above represention in atomic orbital space, we obtain the following relationship:
\begin{equation}
    \Gamma^{AOs} = 
      3\Gamma^{A_1'} + \Gamma^{A_2'}  + 2\Gamma^{A_2''} + 4\Gamma^{E'} + \Gamma^{E''}
\end{equation}

\subsection{Projecting into Vector Space of Irreducible Representation}
Knowing how the vector space of atomic orbitals can be splitted into invariant subspaces 
under symmetry group does not tell us what these subspaces are. 
To find those subspaces, we need to use the projection operator and apply them to atomic basis functions.
\begin{equation}
    P^{\Gamma_i} = \frac{l_n}{h}\sum_R \chi^{\Gamma_i}(R) P_R
\end{equation}
The resulting vector of the projection will belong to the 
subspace that correspond to the required representation. 

In our example, consider only the vector subspace formed by the $s$ orbital centered on three equivalent atoms 
corresponding to the corner of the triangle for simplicity. 
The group representation can be decomposed to $\Gamma^{s3} = \Gamma^{A_1'} + \Gamma^{E'}$ 
with dimension one and two respectively. 
Projecting from each of the $s$ function onto these subspaces, we find that, for $A_1'$:
\begin{align}
    \psi_1^{A_1'} = \frac{1}{3}|s,1\rangle + \frac{1}{3}|s,2\rangle + \frac{1}{3}|s,3\rangle 
    = \psi_2^{A_1'} = \psi_3^{A_1'}
\end{align}
i.e., the projectioned functions span a one dimensional subspace. 
For $E'$ which is a two dimensional subspace, projection gives:
\begin{align}
    \psi_1^{E'} &= \ \ \ \frac{2}{3}|s,1\rangle - \frac{1}{3}|s,2\rangle - \frac{1}{3}|s,3\rangle  \\
    \psi_2^{E'} &= -\frac{1}{3}|s,1\rangle + \frac{2}{3}|s,2\rangle - \frac{1}{3}|s,3\rangle  \\
    \psi_3^{E'} &= -\frac{1}{3}|s,1\rangle - \frac{1}{3}|s,2\rangle + \frac{2}{3}|s,3\rangle 
\end{align}
Noting that $\psi_3^{E'} = - (\psi_1^{E'} + \psi_2^{E'})$, we find that this is indeed a two dimension 
subspace. 

\subsection{Symmetry adopted linear combination}
After separating the invariant subspaces into different irreducible representations, the problem remains to 
find the suitable basis functions. For one dimension represetation, the basis function is unique up to a sign.
However, for irreducible representation larger than one dimension, the choice of basis is not unqie. 
For example, any two of the three functions $\psi_1^{E'}$, $\psi_2^{E'}$ or $\psi_3^{E'}$ gives a basis in $E'$. 

To find basis functions that has symmetry properties, we artifically decrease the symmetry of the system. 
Such a symmetry descend would split irreducible representation into different irreducible representation 
of the subgroup. 
In the previous example, by descending down symmetry from $D_{3h}$ to $C_s$, 
we are able to find better basis functions: group $C_s$ is a subgroup 
of $D_3h$ with only two elements: $\{E, \sigma_v\}$ where $\sigma_v$ is one of the three vertical reflection, 
which we can choose arbitrarily in this case. 
The two dimensional representation $E'$ thus splits under the reduction of symmetry into two one dimensional representation.
Projection of the three functions in $E'$ yield one basis function for each one dimensional representation:
\begin{align}
    \psi_1^{A'} &= \frac{2}{\sqrt{6}} |s,2\rangle - \frac{1}{\sqrt{6}} |s,1\rangle - \frac{1}{\sqrt{6}} |s,3\rangle \\
    \psi_1^{A"} &= \frac{\sqrt{2}}{2} |s,1\rangle - \frac{\sqrt{2}}{2} |s,3\rangle
\end{align}
Being the one dimensional representation of $C_s$, we can identify them to be 
symmetric and anti-symmetric under vertical mirror reflection. Furthermore, the two wavefunctions 
are not allowed to interact with each other since they have different symmetry properties.

\end{document}
