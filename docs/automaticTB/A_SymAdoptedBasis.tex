\documentclass{article}
\usepackage{amssymb, amsmath, amsthm}
\usepackage[margin=1in]{geometry}
\usepackage{verbatim}
\usepackage{graphicx}
\usepackage{hyperref} % \url \href
\usepackage{docmute}

\newtheorem{definition}{Definition}
\newtheorem{theorem}{Theorem}
\DeclareMathOperator{\spn}{Span}

\usepackage[style=chem-acs ,backend=bibtex, sorting=none]{biblatex}
\addbibresource{autoTB.bib}

\begin{document}

\section{Finding the representations and Symmetry adopted basis functions}
\subsection{Molecular orbitals}
Using character table and the orthogonality theorem for characters, we can find the basis functions 
of the Hamiltonians knowing only the symmetry property of the structure. 
We start by finding the vector space of all the participating atomic orbitals and their transformation 
properties under group operation, then decompose the reducible representation to irreducible ones. 
We follow the example of Dresselhaus\cite{dresselhaus_group_2008} and consider the group $D_{3h}$ with 
the following character table \ref{T:ct}
\begin{table}[h!]
    \centering
    \caption{Character table of $D_{3h}$}
    \begin{tabular}{|c|c|c|c|c|c|c|}
                & $E$ & $\sigma_h$ & 2$C_3$ & 2$S_3$ & 2$C_2'$ & 3$\sigma_v$ \\ \hline
         $A_1'$ &  1  &  1         &  1     &  1     &   1     &   1         \\
         $A_2'$ &  1  &  1         &  1     &  1     &  -1     &  -1         \\
         $A_1''$&  1  & -1         &  1     & -1     &   1     &  -1         \\
         $A_2''$&  1  & -1         &  1     & -1     &  -1     &   1         \\
         $E'$   &  2  &  2         &  -1    & -1     &   0     &   0         \\
         $E''$  &  2  & -2         &  -1    &  1     &   0     &   0         \\ \hline
    \end{tabular}
    \label{T:ct}
\end{table}
Suppose we have four atoms, one on the rotation axis and three other that transform to each other 
by three fold rotation, and $s$ and $p$ orbitals on each of the atoms. Indexing each atomic 
centered functions using atomic index $i$ and orbital index $l,m$ as $| i,l,m \rangle$ 
(or $| i,s \rangle, \dots, |i,p_x\rangle$ if convenient), we form 
a vector space with dimension $4\times4 = 16$:
\begin{equation}
    V = \spn(| 1,s \rangle, | 1,p_x \rangle ,\dots, | 4,p_y \rangle, | 4,p_z \rangle)
\end{equation}
We can find the matrix elements of the symmetry group in this space using the following equation:
\begin{equation}
    \rho(g)_{ij} = \langle i | P_g | j \rangle
\end{equation}
where $\rho(g)$ is the matrix in the vector space of atomic basis function and $P_g$ is the 
operators that transform the basis functions. 
After generating the matrix of each symmetry operation, we find the character of the representation
in this atomic orbital basis:
\begin{gather*}
    \chi^{AOs}(E) = 16,\ \chi^{AOs}(\sigma_h) = 8,\ \chi^{AOs}(C_3) = 1\\ 
    \chi^{AOs}(S_3) = -1,\ \chi^{AOs}(C_2') = 0,\ \chi^{AOs}(\sigma_v) = 4
\end{gather*}
The uniqe decomposition of the reducible representation into the irreducible ones are given by:
\begin{gather}
    \chi(C_k) = \sum_{\Gamma_i} a_i \chi^{\Gamma_i}(C_k) \\
    a_i = \frac{1}{h} \sum_{k} N_k \chi^{\Gamma_i}(C_k)^* \chi(C_k)
\end{gather}
where $C_k$ and $N_k$ denote the class of symmetry operation (whose elements have the same character)
and their multiplicity. $h = |G|$ is the order of the group. Decomposing the 
above represention in atomic orbital space, we obtain the following relationship:
\begin{equation}
    \Gamma^{AOs} = 
      3\Gamma^{A_1'} + \Gamma^{A_2'}  + 2\Gamma^{A_2''} + 4\Gamma^{E'} + \Gamma^{E''}
\end{equation}

\subsection{Projecting into invariant subspace}
Knowing who the vector space constructed from the atomic orbitals can be organized into invariant subspaces 
under symmetry group does not tell us what these subspaces are. 
More importantly, to start to reason about the energy of these orbitals, we should start from the approximate
shape of these orbitals.

To start, we first identify what these subspaces are using the projection operation on the atomic basis functions
\begin{equation}
    P^{\Gamma_i} = \frac{l_n}{h}\sum_R \chi^{\Gamma_i}(R) P_R
\end{equation}
where $P^{\Gamma_i}$ is the projection operation onto the representation $\Gamma_i$, $l_n$ is the dimension of the 
representation. 
Operating this projection operator on an arbitrary function in the vector space $V^{AOs}$ produce a vector in the 
subspace that correspond to the required representation. In our previous example, 
if we consider the vector subspace formed by the $s$ orbital centered on three equivalent atoms corresponding to the 
corner of the triangle, the group representation can be decomposed to $\Gamma^{s3} = \Gamma^{A_1'} + \Gamma^{E'}$
with dimension one and two. 
Projecting from each of the $s$ function onto these subspaces, we find that, for $A_1'$:
\begin{align}
    \psi_1^{A_1'} = \frac{1}{3}|s,1\rangle + \frac{1}{3}|s,2\rangle + \frac{1}{3}|s,3\rangle 
    = \psi_2^{A_1'} = \psi_3^{A_1'}
\end{align}
i.e., the projectioned functions span a one dimensional subspace. 
For $E'$ which is a two dimensional subspace, projection gives:
\begin{align}
    \psi_1^{E'} &= \ \ \ \frac{2}{3}|s,1\rangle - \frac{1}{3}|s,2\rangle - \frac{1}{3}|s,3\rangle  \\
    \psi_2^{E'} &= -\frac{1}{3}|s,1\rangle + \frac{2}{3}|s,2\rangle - \frac{1}{3}|s,3\rangle  \\
    \psi_3^{E'} &= -\frac{1}{3}|s,1\rangle - \frac{1}{3}|s,2\rangle + \frac{2}{3}|s,3\rangle 
\end{align}
Noting that $\psi_3^{E'} = - (\psi_1^{E'} + \psi_2^{E'})$, we find that this is indeed a two dimension 
subspace. 

\subsection{Symmetry adopted linear combination}
After separating the invariant subspaces into different irreducible representations, the problem remains to 
find the suitable basis functions. These basis functions should be mutually orthogonality and, at the same times,
exhibit symmetry properties of the group. 
To do so, we need to reduce the symmetry of the system artifically. An irreducible representation become 
reducible when the number of symmetric decreases and the resulting subspaces will be linearly independent 
regarding to the subgroup. As an example, consider the above invariant subspace span by the three functions
$\psi_1^{E'}$, $\psi_1^{E'}$ and $\psi_1^{E'}$. It is clear that these three functions are related to each 
other by rotations, however, they are not orthogonal and each of them do not exhibit clear symmetry property
by themselves. Furthermore, the dimension of this subspace is two and we should eliminate one of the function
from these three, which cannot be done uniquely. 

By descending down symmetry, however, we are able to find better basis functions: group $C_s$ is a subgroup 
of $D_3h$ with only two elements: $\{E, \sigma_v\}$ where $\sigma_v$ is one of the three vertical reflection, 
choosen arbitrarily. 
The two dimensional representation $E'$ splits under the reduction of symmetry into two one dimensional representation.
Projection of the three functions in $E'$ yield one basis function for each one dimensional representation:
\begin{align}
    \psi_1^{A'} &= \frac{2}{\sqrt{6}} |s,2\rangle - \frac{1}{\sqrt{6}} |s,1\rangle - \frac{1}{\sqrt{6}} |s,3\rangle \\
    \psi_1^{A"} &= \frac{\sqrt{2}}{2} |s,1\rangle - \frac{\sqrt{2}}{2} |s,3\rangle
\end{align}
Being the one dimensional representation of $C_s$, we can identify them to be 
symmetric and anti-symmetric under vertical mirror reflection. 

\end{document}
