\documentclass{article}
\usepackage{amssymb, amsmath, amsthm}
\usepackage[margin=1in]{geometry}
\usepackage{verbatim}
\usepackage{graphicx}
\usepackage{hyperref} % \url \href
\usepackage{docmute}
\usepackage{blkarray}

\newtheorem{definition}{Definition}
\newtheorem{theorem}{Theorem}
\DeclareMathOperator{\spn}{Span}

\usepackage[style=chem-acs ,backend=bibtex, sorting=none]{biblatex}
\addbibresource{autoTB.bib}

\begin{document}

\section{METHODS}

In this section, we first describe the tight-binding method, which yield the physical 
properties of the materials in concern. We next show, on the other hand, that using results 
from group theory and symmetry, the free parameters in a tight-binding model can be 
greatly reduced. The higher crystal symmetry, the smaller number of free parameters are 
necessary to describe the electronic structure. This is the key result in this work.

\subsection{AO and Tight-binding Method}
Tight-binding methods is a well known technique to solve the band structure of 
crystalline material with a model Hamiltonian\cite{ziman_principles_1999}. 
It is fast to run and is able to reproduce the DFT calculated band structure accurately 
provided a suitable set of parameters. In this section, we provide a short review of the 
basic methods.

In the tight binding approximation, we write the Hamiltonian in some \emph{local}
basis orbitals:
\begin{equation}
    H_{\mu\nu}(R) = \langle \chi_{R',\mu} | H | \chi_{R'+R,\nu} \rangle = \langle \phi_{0,\mu} | H | \phi_{R,\nu} \rangle
\end{equation}
we use $|\chi_{R,\mu}\rangle$ to denote an electronic states on atom $\mu$ in the cell indexed by lattice vector $R$. 
Due to periodicity, we can arbitrarily choose $R'$ to be zero and arrive at the simplified expression.
The Bloch-like basis functions are given by:
\begin{equation}
    |\psi_{\nu}^k\rangle = \sum_R e^{ik\cdot(R+t_{\nu})} |\phi_{R\nu} \rangle
\end{equation}
and the Hamiltonian matrix at reciprocal vector $\mathbf{k}$ is related to the 
Hamiltonian in real space by:
\begin{equation}
    \label{E:TB}
    H_{\mu\nu}^{\mathbf{k}} = \langle \psi_{\mu}^k | H |\psi_{\nu}^k\rangle = \sum_R e^{ik\cdot(R+t_{\nu}-t_{\mu})} H_{\mu\nu}(R)
\end{equation}
Diagonalizing the square matrix $H_{\mu\nu}^{\mathbf{k}}$ gives the eigen energies $\varepsilon_{\mathbf{k}}$s 
at $\mathbf{k}$ in the tight-binding approximation. 

An important point in the tight-binding approximation is that the interaction matrix elements in real
space should decay rapidly, so that the number of $R$ in the summation \eqref{E:TB} can be greatly 
reduced. In the extreme case, nearest neighbor interactions are considered only. 
The decay depend on the spatial extension of orbitals $|\phi_{R,\nu}\rangle$. 
The valid of this approximate therefore depend on the chemical nature of the bonding. In this work, we use 
atomic orbitals (AOs) as basis functions and assume nearest neighbor interaction only to reduce the 
number of parameters in the model. 

Given set of tight binding parameters $H_{\mu\nu}(R)$, crystal Hamiltonian can be diagonalized at any $k$ points. Therefore, 
band structure and density of states can be obtained straightforwardly. We use tetrahedron method to solve for 
the density of states on a regularly spaced $\mathbf{k}$-grid. 

\subsection{Molecular Orbitals}
In this and following sections, we take a detour to visit the notion of symmetry adopted molecular orbitals. 
For the following, we label atomic orbitals using $|\chi_{\mu}\rangle$ and molecular orbitals $|\psi_i\rangle$. The $i^{th}$ molecular 
orbital can be expressed as a linear combination of atomic orbitals (LCAO): 
\begin{equation}
    \label{E:MO_equation}
    \psi_i\rangle = c_{i\mu} |\chi_{\mu}\rangle + \cdots + c_{i\nu} |\chi_{\nu}\rangle
\end{equation}
As will be described later, we form the linear combinations from a set of symmetry equivalent orbitals, i.e., orbitals 
on equivalent atoms, in a local scope. Therefore, this linear combination is finite and relatively small. For simplicity,
we assume that the atomic orbitals on different atoms have ignorable overlap, then the normalization 
of molecular orbital given by equation \eqref{E:MO_equation} is simply given by the normalization of the 
coefficients $\sum_{\mu}c_{i\mu}^2 = 1$.

The interaction energies in atomic orbitals and molecular orbitals are also related by the coefficients $c_{i\mu}$ as:
\begin{equation}
    \label{E:HAO_HMO_transformation}
    H_{ij} = \sum_{\mu} \sum_{\nu} c_{i\mu} c_{j\nu} H_{\mu\nu}
\end{equation}
where $H_{ij}$ and $H_{\mu\nu}$ are the interaction in terms of molecular orbitals and atomic orbitals. 

\subsection{Symmetry Adapted Molecular Orbitals}
For the moment, we focus on isolated moleculars and discuss the properties of symmetry adopted molecular 
orbitals (symmetry adopted linear combinations, SALCs). Later we discuss the application of similar treatment 
in the case of crystal by focusing on local fragments. 

The symmetry operations of a molecular form a group and symmetry groups provide a way to label different functions 
by its irreducible representations. We provide some more details in the appendix. The conclusion here is 
that the eigenfunctions of the Hamiltonian are also labelled according to the irreducible representations. Functions 
that belong to different irreducible representations have different symmetry properties and their Hamiltonian 
matrix elements will be zero unless the two wavefunctions belong to the same type of irreducible representations:
\begin{equation}
    \langle \psi_i^{\Gamma_a} | H | \psi_j^{\Gamma_b} \rangle = \delta_{ab}
\end{equation}
Furthermore, the energy will be degeneracy for two wavefunctions in the same irreducible:
\begin{equation}
    \langle \psi_1^{\Gamma_a} | H | \psi_1^{\Gamma_a} \rangle = \langle \psi_2^{\Gamma_a} | H | \psi_2^{\Gamma_a} \rangle
\end{equation}
These properties of functions labelled by irreducible representation that we utilize to find the minimum 
interaction parameters. 

In moleculars, we can find a suitable set of basis functions so that single electron states can be 
expressed as a linear combination of them. The set of basis functions form a vector space within which we 
can find SALCs that corresponding to certain irreducible representations. Using groups' character table, 
we can define a projection operator that project out a subspace of a given irreducible representation $\Gamma_i$:
\begin{equation}
    P^{\Gamma_i} = \frac{l_i}{h} \sum_R \chi^{\Gamma_i}(R) P_R
\end{equation}
where $R$ is a group operation, $P_R$ operate on the given functions. 
$l_i$ is the dimension of the representation, $h$ is the order of the group and
$\chi^{\Gamma_i}(R)$ is the character. To start with, we find a spanning basis for the vector space, which 
are simply each individual atomic orbtials we include. Applying the projection operator to each of these 
orbitals yields all wavefunctions that has the same symmetry properties.

We use group $D_{3h}$ and planar molecular AH$_3$ as an example. The 
three dimensional vector space span by three $s$ functions on each H atoms
is given by:
\begin{equation}
    V = \spn(|\chi^s_{1}\rangle, |\chi^s_2\rangle, |\chi^s_3\rangle)
\end{equation}
Using the projection operator, we find two subspaces corresponding to representation $A_1'$ 
and $E'$ with basis:
\begin{align}
    | \psi_1^{A_1'} \rangle &= \frac{\sqrt{3}}{3} (|\chi^s_1\rangle + |\chi^s_2\rangle + |\chi^s_3\rangle) \\
    |\psi_2^{E'}\rangle   &= \frac{\sqrt{6}}{6}( 2|\chi^s_1\rangle - |\chi^s_2\rangle - |\chi^s_3\rangle) \\
    |\psi_3^{E'}\rangle   &= \frac{\sqrt{6}}{6}(- |\chi^s_1\rangle + 2|\chi^s_2\rangle - |\chi^s_3\rangle)
\end{align}
We note that the two basis for representation $E'$ is not orthogonal.
To explore the most from symmetries, we can further split the representation $E'$ into one
that is symmetric and antisymmetric under mirror reflection, by descending symmetry $D_{3h}\to C_s$ (subduction):
\begin{align}
    |\psi_2^{E'_{D_{3h}}\to A'_{C_s}} \rangle &= \frac{2}{\sqrt{6}} |\chi^s_2\rangle - \frac{1}{\sqrt{6}} |\chi^s_1\rangle - \frac{1}{\sqrt{6}} |\chi^s_3\rangle \\
    |\psi_3^{E'_{D_{3h}}\to A''_{C_s}}\rangle &= \frac{\sqrt{2}}{2} |\chi^s_1\rangle - \frac{\sqrt{2}}{2} |\chi^s_3\rangle
\end{align}

We follow the same method for sites in crystals. Interested in local interactions, we focus on one atomic site at a time and its neighbors. 
The symmetry group is therefore given by the site symmetry group of that atomic site. 
There are in total 32 possible site symmetry (point) groups in crystal and all of their character tables can be obtained from Bilbao. 
We can separate all atomic orbitals in a such a local cluster into different symmetry equivalent orbitals and obtain the basis functions 
for each irreducible representations. 
The method to identify site syemmtry and characters for each site symmetry operation are provided in the supplementary. 
To perform symmetry descend to separate degenerate subspaces, we choose the suitable subgroups that are able to separate degenerate 
subspace into representations of different symmetry properties, following the subduction tables 
given in \cite{altmann_point-group_1994}. In principle, the input atomic orbital vectorspace can be splitted into 
one dimensional subspaces. Although there can be multiple functions correspond to the same symmetry type, they belong to distinct 
subspaces and will not mix under symmetry operation. On the other hand, symmetry operation will mix subspaces that are separated 
by symmetry descend, for example, a three fold rotation will mix $|\psi_2^{E'_{D_{3h}}\to A'_{C_s}} \rangle$ with 
$|\psi_3^{E'_{D_{3h}}\to A''_{C_s}}\rangle$, which is the reason for their energy degeneracy.


\subsection{Eliminating symmetry equivalent interactions}
We want to identify the minimum set of interations $H_{\mu\nu}$ which allow the generation of the entire tight-binding Hamiltonian, and we show 
here that the relationship between different $H_{\mu\nu}$s are encoded by $H_{ij}$. This procedure simplifies the size of 
the parameter finding problem as well as ensure the correct symmetry of the Hamiltonian. In the context of machine learning, recent advance in 
rotational invariant models automatically gives some relationship between different $H_{\mu\nu}$s, however, as illustrated by 
the following example, using representation theory, all symmetry relationship can be discovered in a systematic way. 

We consider the simple case of $p_x^l$--$s$--$p_x^r$ interaction, where superscript $l$ and $r$ indicate left and right, in 
a site symmetry of $C_s$ (mirror). Suppose the interaction $s$--$p_x^r$ is $a$, it is apparent to us that the interaction $p_x^l$--$s$ is $-a$. However,
fact is not apparent for an rotational invariant machine learning model, because it is not possible to transform $p_x^l$
into $p_x^r$. 
On the second hand, due to symmetry, the two $p_x$ function can be combined into a symmetric and anti-symmetric part, and 
only the symmetric part interact with the $s$ function. So we have:
\begin{align}
    \label{E:px_s_AO_MO}
    \langle s | H | \psi_A'' \rangle = \frac{\sqrt{2}}{2} \left( \langle s | H | p_x^l \rangle + \langle s | H | p_x^r \rangle\right) = 0 \\
    \langle s | H | \psi_A' \rangle = \frac{\sqrt{2}}{2} \left( \langle s | H | p_x^l \rangle - \langle s | H | p_x^r \rangle\right) = b
\end{align}
which readily give the solution  $\langle s | H | p_x^l \rangle = \langle s | H | p_x^r \rangle = b/2$. 

When parameter $b$ is not known, we will have one equation only, in which case the value of $\langle s | H | p_x^l \rangle$ and 
$\langle s | H | p_x^r \rangle$ can not be solved. However, from the first equation, it is clear that only one of the two 
is free parameter. In more complicated case, free parameters can also be found in this way, since we can find a set of linear equations
\begin{gather}
    \sum_{\mu} \sum_{\nu} c_{1\mu} c_{2\nu} H_{\mu\nu} = H_{12} \\
    \sum_{\mu} \sum_{\nu} c_{1\mu} c_{3\nu} H_{\mu\nu} = H_{13} \\
    \cdots \\
    \sum_{\mu} \sum_{\nu} c_{i\mu} c_{j\nu} H_{\mu\nu} = H_{ij} 
\end{gather}
where many rows are zero on the right side, we can establish a homogeneous equation with number of rows $m$ less than number of 
unknown $n$. In such case, we will have in general $m-n$ free variables. To find these variable, we transform the 
coefficient matrix $C_{m\times n}$ into a row echelon form and each column without a leading variables would correspond to 
a free parameters. In this work, we use Gaussian elimination to bring the coefficient matrix to row echelon form.

Once suitable values are assigned to each of the free variables, values of all the unknowns can be found by appending to the homogeneous 
equation with rows, each has the free variables on the left and their corresponding on the right. This procedure produce a full rank 
coefficient and an linear equation system with unique solution. 

\subsection{Form of the interaction Hamiltonian in MOs}
We mentioned previously that interactions are forbidden between MOs belonging to different 
irreducible representations. Here, we describe in detail the form of interaction matrix. 
We assume that all MOs are of the form that utilize symmetry, denoted in the form 
$|\psi_i^{\Gamma_1\to \cdots \to \Gamma_n} \rangle$. As shown in the previous section, using crystallographic 
point groups and subduction, all MOs can be separated into one dimensional subspace. However, we note that 
the first classification of irreducible representation $\Gamma_1$ is the most important (main irreducible representation), since it correspond
to the true symmetry of the structure and the degeneracy of energy. 
The rest symbols $\to \cdots \to \Gamma_n$ of subduction aim to separate basis of the degenerate subspaces and 
ensure that the basis are orthogonal. 

The interaction matrix in MOs can then be separated into different diagonal blocks, each blocks are of the form $a\cdot \mathbf{I}$ if 
the interacting MOs correspond to the same main irreducible representation, zero otherwise.
For example, in the case of AH$_3$ molecular.
\begin{equation}
    H_{ij} = \begin{blockarray}{cccccccc}
        & |s^{A_1'}\rangle & |p_z^{A_2''}\rangle & |p_x^{E'}\rangle & |p_y^{E'}\rangle 
        & |S^{A_1'}\rangle & |S^{E'}_{p_x}\rangle & |S^{E'}_{p_y}\rangle \\
        \begin{block}{c(ccccccc)}
        \langle s^{A_1'}    |  & a & 0 & 0 & 0 & d & 0 & 0 \\
        \langle p_z^{A_2''} |  & 0 & b & 0 & 0 & 0 & 0 & 0 \\
        \langle p_x^{E'}    |  & 0 & 0 & c & 0 & 0 & e & 0 \\
        \langle p_y^{E'}    |  & 0 & 0 & 0 & c & 0 & 0 & e \\
        \end{block}
        \end{blockarray}
\end{equation}

\subsection{Finding independent parameters}
Given a linear equation system \eqref{E:HAO_HMO_transformation} where a large number of $H_{ij}$ are zeros,
the number of free parameters are greatly reduced. The set of equations with $H_{ij} = 0$ constitute an 
underdetermined equation system which constrain the solution of $H_{\mu\nu}$s. For example, the first 
equation in \eqref{E:px_s_AO_MO} constrains the value of $\langle s | H | p_x^l \rangle$ to be negative of 
that of $\langle s | H | p_x^r \rangle$, eventhough the value itself is given by the second equation. 
Similarly, the constrains given by the equation with $H_{ij} = 0$ limits the number of free parameters 
to be the same as the number of rows where $H_{ij} \neq 0$. 

Therefore, the problem of finding the free parameter $H_{\mu\nu}$ is thus a problem to find the homogeneous 
equation system and then find the free parameters. This can be done by, for example, LU fractization of the 
non-square matrix $A$.

\subsection{Machine Learning of $H_{\mu\nu}$}
 

\end{document}
