\documentclass{article}
\usepackage{amssymb, amsmath, amsthm}
\usepackage[margin=1in]{geometry}
\usepackage{verbatim}
\usepackage{graphicx}
\usepackage{hyperref} % \url \href
\usepackage{docmute}
\usepackage{blkarray}

\newtheorem{definition}{Definition}
\newtheorem{theorem}{Theorem}
\DeclareMathOperator{\spn}{Span}

\usepackage[style=chem-acs ,backend=bibtex, sorting=none]{biblatex}
\addbibresource{autoTB.bib}

\begin{document}

\section{METHODS}

\subsection{AO and Tight-binding Method}
Tight-binding methods is a well known technique to solve the band structure of 
crystalline material with a model Hamiltonian\cite{ziman_principles_1999}. 
It is fast to run and is able to reproduce the DFT calculated band structure accurately 
provided a suitable set of parameters. In this section, we provide a short review of the 
basic methods.

In the tight binding approximation, we write the Hamiltonian in some \emph{local}
basis orbitals:
\begin{equation}
    H_{ij}(R) = \langle \phi_{R',i} | H | \phi_{R'+R,j} \rangle = \langle \phi_{0,i} | H | \phi_{R,j} \rangle
\end{equation}
where index $R'$, $R$ are lattice vectors. Due to periodicity, we can arbitrarily choose $R'$ to be zero
and arrive at the simplified expression.
The Bloch-like basis functions are given by:
\begin{equation}
    |\psi_j^k\rangle = \sum_R e^{ik\cdot(R+t_j)} |\phi_{Rj} \rangle
\end{equation}
and the Hamiltonian matrix at reciprocal vector $\mathbf{k}$ is related to the 
Hamiltonian in real space by:
\begin{equation}
    \label{E:TB}
    H_{ij}^{\mathbf{k}} = \langle \psi_i^k | H |\psi_j^k\rangle = \sum_R e^{ik\cdot(R+t_j-t_i)} H_{ij}(R)
\end{equation}
Diagonalizing the square matrix $H_{ij}^{\mathbf{k}}$ gives the eigen energies $\varepsilon_{i,\mathbf{k}}$ 
at $\mathbf{k}$ in the tight-binding approximation. 

An important point in the tight-binding approximation is that the interaction matrix elements in real
space should decay rapidly, so that the number of $R$ in the summation \eqref{E:TB} can be greatly 
reduced. In the extreme case, nearest neighbor interactions are considered only. 
The decay depend on the spatial extension of orbitals $|\phi_{R,j}\rangle$. 
The valid of this approximate therefore depend on the chemical nature of the bonding. In this work, we use 
atomic orbitals (AOs) as basis functions and assume nearest neighbor interaction only to reduce the 
number of parameters in the model. 

Given set of tight binding parameters $H_{ij}(R)$, crystal Hamiltonian can be diagonalized at any k points. Therefore, 
band structure and density of states can be obtained straightforwardly. We use tetrahedron method to solve for 
the density of states on a regularly spaced $\mathbf{k}$-grid. 

\subsection{Symmetry Adopted Molecular Orbitals}
For the following, we label atomic orbitals using $|\chi\rangle$ and molecular orbitals $|\psi\rangle$. The $i^{th}$ molecular 
orbital can be expressed as a linear combination of atomic orbitals (LCAO): 
\begin{equation}
    \psi_i\rangle = c_{i\mu} |\chi_{\mu}\rangle + \cdots + c_{i\nu} |\chi_{\nu}\rangle
\end{equation}
Assume that each atomic orbitals are normalized and orthogonal if they are on the same atom. Further assume 
that atomic orbitals at different sites have ignorable overlap, then the normalization 
of $\psi_i\rangle$ is simply given by the normalization of the coefficients. 
The Hamiltonian in molecular $H_{ij} = \langle \psi_i | H | \psi_j \rangle$ is then:
\begin{equation}
    \label{E:HAO_HMO_transformation}
    H_{ij} = \sum_{\mu} \sum_{\nu} c_{i\mu} c_{j\nu} H_{\mu\nu}
\end{equation}
where $H_{\mu\nu} = \langle \chi_{\mu} | H | \chi_{\nu} \rangle$ is the interaction in atomic orbitals.

According to the result from application of group theory in quantum chemistry, it can be 
shown that the eigenfunction of the Hamiltonian occupy differrent subspaces, labelled by the 
irreducible representation of the symmetry group. Furthermore, given a vector space consists 
of a set of basis functions, it is possible to find these subspaces using groups' character table
by projection operator into representation $\Gamma_i$
\begin{equation}
    P^{\Gamma_i} = \frac{l_i}{h} \sum_R \chi^{\Gamma_i}(R) P_R
\end{equation}
where $R$ is a group operation, $P_R$ operate on the given functions. 
$l_i$ is the dimension of the representation, $h$ is the order of the group and
$\chi^{\Gamma_i}(R)$ is the character. 

We use group $D_{3h}$ and planar molecular AH$_3$ as an example. The 
three dimensional vector space span by three $s$ functions on each H atoms
is given by:
\begin{equation}
    V = \spn(|\chi^s_{1}\rangle, |\chi^s_2\rangle, |\chi^s_3\rangle)
\end{equation}
Using the projection operator, we find two subspaces corresponding to representation $A_1'$ 
and $E'$ with basis:
\begin{align}
    | \psi_1^{A_1'} \rangle &= \frac{\sqrt{3}}{3} (|\chi^s_1\rangle + |\chi^s_2\rangle + |\chi^s_3\rangle) \\
    |\psi_2^{E'}\rangle   &= \frac{\sqrt{6}}{6}( 2|\chi^s_1\rangle - |\chi^s_2\rangle - |\chi^s_3\rangle) \\
    |\psi_3^{E'}\rangle   &= \frac{\sqrt{6}}{6}(- |\chi^s_1\rangle + 2|\chi^s_2\rangle - |\chi^s_3\rangle)
\end{align}
We note that the two basis for representation $E'$ is not orthogonal.
To explore the most from symmetries, we can further split the representation $E'$ into one
that is symmetric and antisymmetric under mirror reflection, by descending symmetry $D_{3h}\to C_s$:
\begin{align}
    |\psi_2^{E'_{D_{3h}}\to A'_{C_s}} \rangle &= \frac{2}{\sqrt{6}} |\chi^s_2\rangle - \frac{1}{\sqrt{6}} |\chi^s_1\rangle - \frac{1}{\sqrt{6}} |\chi^s_3\rangle \\
    |\psi_3^{E'_{D_{3h}}\to A''_{C_s}}\rangle &= \frac{\sqrt{2}}{2} |\chi^s_1\rangle - \frac{\sqrt{2}}{2} |\chi^s_3\rangle
\end{align}

We follow the same method for sites in crystals. Interested in local interactions, we focus on one atomic site at a time and its neighbors. 
The symmetry group is therefore given by the site symmetry group of that atomic site. 
There are in total 32 possible site symmetry (point) groups in crystal and all of their character tables can be obtained from Bilbao. 
The method to identify site syemmtry and characters for each site symmetry operation are provided in the supplementary. 
To perform symmetry descend to separate degenerate subspaces, we choose the suitable subgroups that are able to separate degenerate 
subspace into representations of different symmetry properties, following the subduction tables 
given in \cite{altmann_point-group_1994}. In principle, all representation can be split into one dimensional subspace. 
For detail, we refer to supplementary information.

\subsection{Using MOs instead of AOs}
Our goal in this work is concerned with the generation of a nearest neighbor tight binding model, in which 
the parameters are Hamiltonian matrix elements between AOs $H_{\mu\nu}$. However, we choose to find 
the Hamiltonian matrix elements between MOs $H_{ij}$ first and recover $H_{\mu\nu}$s by inverting the linear equation \eqref{E:HAO_HMO_transformation}.
The symmetry adopted MOs lead to two following properties of the 
result and simplifies the problem:
\begin{enumerate}
    \item automatic symmetrization of tight-binding model, and 
    \item utilize crystal symmetry to remove the number of parameters ML required to predict.
\end{enumerate}
The first property is straightforward. Since MOs are symmetries, interaction between AOs would also be symmetrized. The simplification come 
from the fact that interactions are automatically forbidden by symmetry between MOs in different representations. Therefore, the non-zero interaction 
in MOs is sparse, especially when the crystal symmetry is high. 

To illustrate these two point, we use a simple case of $p_x^l$--$s$--$p_x^r$ interaction, where superscript $l$ and $r$ indicate left and right, in 
a site symmetry of $C_s$ (mirror). Suppose the interaction $s$--$p_x^r$ is $a$, it is apparent to us that the interaction $p_x^l$--$s$ is $-a$. However,
this is not apparent for a machine learning model, even if it is rotational invariant or equivariant, because it is not possible to transform $p_x^l$
into $p_x^r$. 

On the second hand, due to symmetry, the two $p_x$ function can be combined into a symmetric and anti-symmetric part, and 
only the symmetric part interact with the $s$ function. So we have:
\begin{align}
    \label{E:px_s_AO_MO}
    \langle s | H | \psi_A'' \rangle = \frac{\sqrt{2}}{2} \left( \langle s | H | p_x^l \rangle + \langle s | H | p_x^r \rangle\right) = 0 \\
    \langle s | H | \psi_A' \rangle = \frac{\sqrt{2}}{2} \left( \langle s | H | p_x^l \rangle - \langle s | H | p_x^r \rangle\right) = b
\end{align}
which readily give the solution  $\langle s | H | p_x^l \rangle = \langle s | H | p_x^r \rangle = b/2$.

\subsection{Transformation between AOs and MOs}
As stated previous, we treat each atom in the unit cell on at a time to find relevant tight binding matrix element $\langle \chi_{i} | H | \chi^{NN}_{R,j} \rangle$.
where the $\langle \chi_{i} |$ is an atomic orbtial on the atom in focus and $| \chi^{NN}_{R,j} \rangle$ are the set of atomic orbitals in the nearest neighbors (NN)
including the center atom. 
The selection of nearest neighbors atoms can either be the first shell or the Voronoi bonding, depending on the simplification level desired. 

Next, we find the symmetry adopted linear combinations (SALCs) composed of $N_c$ AOs of the center atom and $N_{nn}$ of all the neighboring atoms. 
There are $N_c\times (N_c + N_{nn})$ matrix elements $H_{\mu\nu}$ to be found as tight binding matrix element and the same number of $H_{ij}$ for 
the interaction between molecular orbitals. We find a linear equation system given by equation \eqref{E:HAO_HMO_transformation} and 
obtain the solution once the interaction energy between pairs MOs are known.

\subsection{Form of the interaction Hamiltonian in MOs}
We mentioned previously that interactions are forbidden between MOs belonging to different 
irreducible representations. Here, we describe in detail the form of interaction matrix. 
We assume that all MOs are of the form that utilize symmetry, denoted in the form 
$|\psi_i^{\Gamma_1\to \cdots \to \Gamma_n} \rangle$. As shown in the previous section, using crystallographic 
point groups and subduction, all MOs can be separated into one dimensional subspace. However, we note that 
the first classification of irreducible representation $\Gamma_1$ is the most important (main irreducible representation), since it correspond
to the true symmetry of the structure and the degeneracy of energy. 
The rest symbols $\to \cdots \to \Gamma_n$ of subduction aim to separate basis of the degenerate subspaces and 
ensure that the basis are orthogonal. 

The interaction matrix in MOs can then be separated into different diagonal blocks, each blocks are of the form $a\cdot \mathbf{I}$ if 
the interacting MOs correspond to the same main irreducible representation, zero otherwise.
For example, in the case of AH$_3$ molecular.
\begin{equation}
    H_{ij} = \begin{blockarray}{cccccccc}
        & |s^{A_1'}\rangle & |p_z^{A_2''}\rangle & |p_x^{E'}\rangle & |p_y^{E'}\rangle 
        & |S^{A_1'}\rangle & |S^{E'}_{p_x}\rangle & |S^{E'}_{p_y}\rangle \\
        \begin{block}{c(ccccccc)}
        \langle s^{A_1'}    |  & a & 0 & 0 & 0 & d & 0 & 0 \\
        \langle p_z^{A_2''} |  & 0 & b & 0 & 0 & 0 & 0 & 0 \\
        \langle p_x^{E'}    |  & 0 & 0 & c & 0 & 0 & e & 0 \\
        \langle p_y^{E'}    |  & 0 & 0 & 0 & c & 0 & 0 & e \\
        \end{block}
        \end{blockarray}
\end{equation}
In our work, the parameters $a \cdots e$ are to be found using machine learning methods, for example. 

\subsection{Finding independent parameters}
Given a linear equation system \eqref{E:HAO_HMO_transformation} where a large number of $H_{ij}$ are zeros,
the number of free parameters are greatly reduced. The set of equations with $H_{ij} = 0$ constitute an 
underdetermined equation system which constrain the solution of $H_{\mu\nu}$s. For example, the first 
equation in \eqref{E:px_s_AO_MO} constrains the value of $\langle s | H | p_x^l \rangle$ to be negative of 
that of $\langle s | H | p_x^r \rangle$, eventhough the value itself is given by the second equation. 
Similarly, the constrains given by the equation with $H_{ij} = 0$ limits the number of free parameters 
to be the same as the number of rows where $H_{ij} \neq 0$. 

Therefore, the problem of finding the free parameter $H_{\mu\nu}$ is thus a problem to find the homogeneous 
equation system and then find the free parameters. This can be done by, for example, LU fractization of the 
non-square matrix $A$.

\subsection{Machine Learning of $H_{\mu\nu}$}
 

\end{document}
