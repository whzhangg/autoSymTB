\documentclass{article}
\usepackage{amssymb, amsmath, amsthm}
\usepackage[margin=1in]{geometry}
\usepackage{verbatim}
\usepackage{graphicx}
\usepackage{hyperref} % \url \href
\usepackage{docmute}

\newtheorem{definition}{Definition}
\newtheorem{theorem}{Theorem}
\DeclareMathOperator{\spn}{Span}

\usepackage[style=chem-acs ,backend=bibtex, sorting=none]{biblatex}
\addbibresource{autoTB.bib}

\begin{document}

\section{METHODS}

\subsection{AO and Tight-binding Method}
Tight-binding methods is a well known technique to solve the band structure of 
crystalline material with a model Hamiltonian\cite{ziman_principles_1999}. 
It is fast to run and is able to reproduce the DFT calculated band structure accurately 
provided a suitable set of parameters. In this section, we provide a short review of the 
basic methods.

In the tight binding approximation, we write the Hamiltonian in some \emph{local}
basis orbitals:
\begin{equation}
    H_{ij}(R) = \langle \phi_{R',i} | H | \phi_{R'+R,j} \rangle = \langle \phi_{0,i} | H | \phi_{R,j} \rangle
\end{equation}
where index $R'$, $R$ are lattice vectors. Due to periodicity, we can arbitrarily choose $R'$ (0) 
and arrive at the simplified expression.
The Bloch-like basis functions are then:
\begin{equation}
    |\psi_j^k\rangle = \sum_R e^{ik\cdot(R+t_j)} |\phi_{Rj} \rangle
\end{equation}
and the Hamiltonian matrix at reciprocal vector $\mathbf{k}$ is related to the 
Hamiltonian in real space by:
\begin{equation}
    \label{E:TB}
    H_{ij}^{\mathbf{k}} = \langle \psi_i^k | H |\psi_j^k\rangle = \sum_R e^{ik\cdot(R+t_j-t_i)} = H_{ij}(R)
\end{equation}
Diagonalizing the square matrix $H_{ij}^{\mathbf{k}}$ gives the eigen energies $\epsilon_{i,\mathbf{k}}$ 
at $\mathbf{k}$ in the tight-binding approximation. 

An important point in the tight-binding approximation is that the interaction matrix elements in real
space should decay rapidly, so that the number of $R$ in the summation \eqref{E:TB} can be greatly 
reduced. In the extreme case, nearest neighbor interactions are considered only. 
The decay depend on the spatial extension of orbitals $|\phi_{R,j}\rangle$. 
The valid of this approximate depend on the chemical nature of the bonding. In this work, we use 
atomic orbitals (AOs) as basis functions and assume nearest neighbor interaction only to reduce the 
number of parameters in the model. 

Given set of tight binding parameters $H_{ij}(R)$, crystal Hamiltonian can be diagonalized at any k points. Therefore, 
band structure and density of states can be obtained straightforwardly. We use tetrahedron method to solve for 
the density of states on a regularly spaced $\mathbf{k}$-grid. 

\subsection{Symmetry Adopted Molecular Orbitals}
According to the result from application of group theory in quantum chemistry, it can be 
shown that the eigenfunction of the Hamiltonian occupy differrent subspaces, labelled by the 
irreducible representation of the symmetry group. Furthermore, given a vector space consists 
of a set of basis functions, it is possible to find these subspaces using groups' character table
by projection operator into representation $\Gamma_i$
\begin{equation}
    P^{\Gamma_i} = \frac{l_i}{h} \sum_R \chi^{\Gamma_i}(R) P_R
\end{equation}
where $R$ is a group operation, $P_R$ operate on the given functions. 
$l_i$ is the dimension of the representation, $h$ is the order of the group and
$\chi^{\Gamma_i}(R)$ is the character. 

We use group $D_{3h}$ and planar molecular AH$_3$ as an example. The 
three dimensional vector space span by three $s$ functions on each H atoms
is given by:
\begin{equation}
    V = \spn(|s_1\rangle, |s_2\rangle, |s_3\rangle)
\end{equation}
Using the projection operator, we find two subspaces corresponding to representation $A_1'$ 
and $E'$:
\begin{align}
    | \psi^{A_1'} \rangle &= \frac{\sqrt{3}}{3} (|s_1\rangle + |s_2\rangle + |s_3\rangle) \\
    |\psi_1^{E'}\rangle   &= \frac{\sqrt{6}}{6}( 2|s_1\rangle - |s_2\rangle - |s_3\rangle) \\
    |\psi_2^{E'}\rangle   &= \frac{\sqrt{6}}{6}(- |s_1\rangle + 2|s_2\rangle - |s_3\rangle)
\end{align}

To explore the most from symmetries, we can further split the representation $E'$ into one
that is symmetric and antisymmetric under mirror reflection, by descending symmetry $D_{3h}\to C_s$:
\begin{align}
    |\psi^{E'_{D_{3h}}\to A'_{C_s}} \rangle &= \frac{2}{\sqrt{6}} |s,2\rangle - \frac{1}{\sqrt{6}} |s,1\rangle - \frac{1}{\sqrt{6}} |s,3\rangle \\
    |\psi^{E'_{D_{3h}}\to A''_{C_s}}\rangle &= \frac{\sqrt{2}}{2} |s,1\rangle - \frac{\sqrt{2}}{2} |s,3\rangle
\end{align}.

We follow the same method for sites in crystals. To consider the interaction between MOs of one atom and its neighbors, 
we need to form the symmetry adopted MOs. We first obtain the site symmetry group of the atom in focus. The 
method to identify site syemmtry and characters for each site symmetry operation are provided in the supplementary. 
There are in total 32 possible site symmetry (point) groups in crystal and all of their character tables are obtained from Bilbao. 
To perform symmetry descend to separate degenerate subspaces, we choose the suitable subgroups that are able to separate degenerate 
subspace into representations of different symmetry properties, following the subduction tables 
given in \cite{altmann_point-group_1994}. In principle, all representation can be split into one dimensional subspace. 
For detail, we refer to supplementary information.

\subsection{Using MOs instead of AOs}
Our goal in this work is concerned with the generation of a nearest neighbor tight binding model, in which 
the parameters are Hamiltonian matrix elements between AOs. However, we choose to find 
the Hamiltonian matrix elements between MOs $\langle \psi^{MO}_1 | H | \psi^{MO}_2 \rangle$ first, and \emph{then} find the corresponding 
interaction matrix elements in AOs $\langle \phi^{AO}_1 | H | \phi^{AO}_2 \rangle$. The reasons here are two:
\begin{enumerate}
    \item automatic symmetrization of tight-binding model, and 
    \item utilize crystal symmetry to remove the number of parameters ML required to predict.
\end{enumerate}
The first reason is straightforward. Since MOs are symmetries, interaction between AOs would also be symmetrized. The second reason come 
from the fact that interactions are automatically forbidden by symmetry between MOs in different representations. Therefore, the non-zero interaction 
in MOs is sparse, especially when the crystal symmetry is high. 

To illustrate these two point, we use a simple case of $p_x^l$--$s$--$p_x^r$ interaction, where superscript $l$ and $r$ indicate left and right, in 
a site symmetry of $C_s$ (mirror). Suppose the interaction $s$--$p_x^r$ is $a$, it is apparent to us that the interaction $p_x^l$--$s$ is $-a$. However,
this is not apparent for a machine learning model, even if it is rotational invariant or equivariant, because it is not possible to transform $p_x^l$
into $p_x^r$. 

On the second hand, due to symmetry, the two $p_x$ function can be combined into a symmetric and anti-symmetric part, and 
only the symmetric part interact with the $s$ function. So we have:
\begin{align}
    \label{E:px_s_AO_MO}
    \langle s | H | \psi_A'' \rangle = \frac{\sqrt{2}}{2} \left( \langle s | H | p_x^l \rangle + \langle s | H | p_x^r \rangle\right) = 0 \\
    \langle s | H | \psi_A' \rangle = \frac{\sqrt{2}}{2} \left( \langle s | H | p_x^l \rangle - \langle s | H | p_x^r \rangle\right) = b
\end{align}
which readily give the solution  $\langle s | H | p_x^l \rangle = \langle s | H | p_x^r \rangle = b/2$.

\subsection{Transformation between AOs and MOs}
The transform between AOs and MOs are examplified by Equation \eqref{E:px_s_AO_MO}. In this section, we describe in more detail the treatments
in crystals.

The required parameters in a nearest neighbor tight-binding model is the matrix elements: $\langle \phi_{i} | H | \phi^{NN}_{R,j} \rangle$,
where $i$ and $j$ are index of atoms in the unit cell, superscript NN indicate that the atomic $j$ in the $R^{th}$ cell that belong to the 
set of nearest neighbor atoms. We tackle this problem for each atom $i$ in the unit cell by finding, first of all, the set of neighbor 
atoms. They are not necessary in the unit cell of $i$ itself. For concreteness, suppose there are $N_i$ atomic orbitals on the atom $i$, and 
$N_neighbor \times N_j$ atomic orbitals. We construct $N_i$ symmetrized MOs for atom $i$ and $N_neighbor \times N_j$ symmetrized MOs for 
the equivalent atoms $j$. There are in total $N_i \times N_neighbor \times N_j$ MO interactions, with the form 
$\langle \psi_{i} | H | \psi^{NN}_{js} \rangle$. Subsituting MOs by a linear combination of AOs, we would obtain a linear equation system 
in the form:
\begin{gather}
    a_{11} \langle \phi_{i}^1 | H | \phi^{NN,1}_{R,j} \rangle + \cdots + a_{1n} \langle \phi_{i}^n | H | \phi^{NN,n}_{R,j} \rangle 
     = \langle \psi_{i}^1 | H | \psi^{NN,1}_{js} \rangle  \\
     \cdots \\
     a_{n1} \langle \phi_{i}^1 | H | \phi^{NN,1}_{R,j} \rangle + \cdots + a_{nn} \langle \phi_{i}^n | H | \phi^{NN,n}_{R,j} \rangle 
      = \langle \psi_{i}^n | H | \psi^{NN,n}_{js} \rangle 
\end{gather}
which can be solved to provide the interaction matrix elements between the nearest neighbor atomic orbitals

\subsection{Form of the interaction Hamiltonian in MOs}
We mentioned previously that interactions are forbidden between MOs belonging to different 
irreducible representations. Here, we describe in detail the form of interaction matrix. 

\end{document}
