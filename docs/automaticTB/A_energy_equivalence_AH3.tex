\documentclass{article}
\usepackage{amssymb, amsmath, amsthm}
\usepackage[margin=1in]{geometry}
\usepackage{verbatim}
\usepackage{graphicx}
\usepackage{hyperref} % \url \href
\usepackage{docmute}

\newtheorem{definition}{Definition}
\newtheorem{theorem}{Theorem}
\DeclareMathOperator{\spn}{Span}

\usepackage[style=chem-acs ,backend=bibtex, sorting=none]{biblatex}
\addbibresource{autoTB.bib}

\begin{document}

\section{Equivalent Interaction for AH3}
In this section, we show that, for the case of AH3, we have:
\begin{equation}
    \label{E:equivalent_AH3}
    \langle \psi^{E'\to A'}_{A,p} | H | \psi^{E'\to A'}_{H,s} \rangle = \langle \psi^{E'\to A''}_{A,p} | H | \psi^{E'\to A''}_{H,s} \rangle
\end{equation}
The following three interactions are related by a three fold rotation:
\begin{equation}
    \langle p_x | H | s_1 \rangle = \langle -\frac{1}{2}p_x + \frac{\sqrt{3}}{2}p_y | H | s_2 \rangle 
    = \langle -\frac{1}{2}p_x - \frac{\sqrt{3}}{2}p_y | H | s_3 \rangle
\end{equation}
where the three combinations of the $p$ orbitals on atom A points towards the three $s$ functions on the H atoms. We also have 
\begin{equation}
    \langle p_y | H | s_1 \rangle = \langle -\frac{\sqrt{3}}{2}p_x - \frac{1}{2}p_y | H | s_2 \rangle 
    = \langle \frac{\sqrt{3}}{2}p_x - \frac{1}{2}p_y | H | s_3 \rangle = 0
\end{equation}
where in this case, the $p$ orbitals points 90 degree to the $s$ functions and the interactions cancels.
Solving the above two equations, we have the following relationship:
\begin{gather}
    \label{E:AH3_relationship}
    \langle p_y | H | s_1 \rangle = 0 \\
    \langle p_x | H | s_1 \rangle = -2 \langle p_x | H | s_2 \rangle = -2 \langle p_x | H | s_3 \rangle = 
    \frac{\sqrt{3}}{3} \langle p_y | H | s_2 \rangle = - \frac{\sqrt{3}}{3} \langle p_y | H | s_3 \rangle
\end{gather}
Using there relationships, we can find:
\begin{align}
   \frac{\sqrt{6}}{6} \langle p_x | H | 2 s_1 -  s_2 -  s_3 \rangle &= \frac{\sqrt{6}}{2} \langle p_x | H | s_1 \rangle \\
   \frac{\sqrt{2}}{2} \langle p_y | H | s_2 -  s_3 \rangle &= \frac{\sqrt{6}}{2} \langle p_x | H | s_1 \rangle 
\end{align}
showing that the two interaction given in equation \eqref{E:equivalent_AH3} indeed have the same value. 
We comment that the relationship given in equation \eqref{E:AH3_relationship} are the same as the relationship 
recoveried using the symmetry restrains and solving the homogeneous equations, as mentioned in the 
article. However, the method here is rather ad-hoc. For example, it is not clear why the values such as 
$\langle \frac{\sqrt{3}}{2}p_x - \frac{1}{2}p_y | H | s_3 \rangle$ is zero without carrying out intergral (although 
it is apparent by visual inspectations). The constrains solved from interactions between symmetrized molecular 
functions, however, are systematic. 

\end{document}