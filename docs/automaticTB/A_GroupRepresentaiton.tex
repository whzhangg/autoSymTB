\documentclass{article}
\usepackage{amssymb, amsmath, amsthm}
\usepackage[margin=1in]{geometry}
\usepackage{verbatim}
\usepackage{graphicx}
\usepackage{hyperref} % \url \href
\usepackage{docmute}

\newtheorem{definition}{Definition}
\newtheorem{theorem}{Theorem}
\DeclareMathOperator{\spn}{Span}

\usepackage[style=chem-acs ,backend=bibtex, sorting=none]{biblatex}
\addbibresource{autoTB.bib}

\begin{document}

\section{Groups, representations and basis}
\subsection{Groups and representations}
The symmetry operation that map a molecular onto itself form a group $G$. From a group,
we define a representation of group $G$ as a mapping:
\begin{definition}
    A linear \emph{representation} of a group $G$ on a vector space $V$ is a homomorphism:
    \[\rho\colon G \to \text{GL}(V)\]
    where homomorphism is a mapping that satisfies $\rho(g_1g_2) = \rho(g_1)\rho(g_2)$.
\end{definition}
Representations are not unique: a change of basis leads to an equivalent representation 
and the direct sum of two representation is again a representation. However, representations
can be uniquely reduced to so called \emph{irreducible representations}.
We define the notion of \emph{invariant vector space}:
\begin{definition}
    For any vector $v$ in $V$, for any $g\in G$, if we have $\rho(g)v \in V$, then $V$
    is an invariant vector space.
\end{definition} 
For a representation with vector space $V$, $V$ is an invariant to the group $G$. 
If there exist a vector space $W\subset V$ which is also invariant to $G$, then we call $W$
an \emph{invariant subspace}. 
Furthermore, there exist another invariant subspace $U$ of $V$ so that $V$ is given by a direct sum:
\begin{equation}
    V = U \oplus W
\end{equation}
By splitting the vector space $V$ into invariant subspaces until it is not possible, we find the 
\emph{irreducible representation} by the reducible representation on $V$, and the matrix form 
of the reducible representation can be written as the block diagonal form
\begin{equation}
    \rho^{\Gamma}(g) = \left(  
        \begin{matrix}
            \rho^{\Gamma_1} (g) & 0 & \cdots\\
            0 & \rho^{\Gamma_2} (g) & \\
            \vdots &  & \ddots
        \end{matrix}
    \right)
\end{equation}
where $\Gamma$ denote the reducible representation and $\Gamma_i\ (i = 1, \dots)$ index 
the irreducible representations.

\subsection{Basis functions}
For each invariant subspaces corresponding to a representation $\Gamma_n$, we write its basis 
functions using the braket notation $|\Gamma_n,j\rangle$. For a irreducible representation with 
dimension $N$, we have $N$ such basis functions and any vectors in this invariant subspace 
can be written as a linear combination:
\begin{equation}
    |\phi^{\Gamma_n}\rangle = \sum_{i = 1}^N c_i |\Gamma_n,i\rangle
\end{equation} 
Due to the definition of invariant subspace, the matrix representation of $G$ in the invariant 
subspace will have the form:
\begin{equation}
    \rho^{\Gamma_n}(g)_{ij} = \langle \Gamma_n,i | P_g | \Gamma_n,j \rangle
\end{equation}
where $P_g$ is the symmetry operation. 
For reducible representations $\Gamma_{r}$, its vector subspace is a direct sum of all the subspace 
of irreducible representations it contains. Therefore, basis functions from irreducible 
subspaces together form the basis functions for the reducible representation.

The basis functions of the representation of the symmetry group is closely connected to quantum mechanics.
Since the Hamiltonian of the system is invariant to the symmetry operation, we have the commutation 
relationship:
\begin{equation}
    [P_g,H] = 0
\end{equation}
for any symmetry operation $g$. As a consequence, if $|\phi_{\varepsilon_i}\rangle$ an eigenfunction of $H$
with eigenvalue $\varepsilon_i$, $P_g|\phi_{\varepsilon_i}\rangle$ will also be an eigenfunction 
with degenerate eigenvalues. Since $|\phi_{\varepsilon_i}\rangle$ and $P_g|\phi_{\varepsilon_i}\rangle$
together will define some invariant space. Therefore, it is possible to classify the eigenfunctions of the 
Hamiltonian according to the subspaces which they occupy, and the set of degenerate eigenfunctions are 
related by symmetry operation. 

\end{document}
