\documentclass{article}
\usepackage{amssymb, amsmath, amsthm}
\usepackage[margin=1in]{geometry}
\usepackage{verbatim}
\usepackage{graphicx}
\usepackage{hyperref} % \url \href
\usepackage{docmute}

\newtheorem{definition}{Definition}
\newtheorem{theorem}{Theorem}
\DeclareMathOperator{\spn}{Span}

\usepackage[style=chem-acs ,backend=bibtex, sorting=none]{biblatex}
\addbibresource{autoTB.bib}

\begin{document}

\section{Group Representations and Basis}
In this section we provide a brief description of some group theoretical results
which we used. The main reference is J.P. Serre 1996\cite{serre_linear_1996}
and Dresselhaus et al. 2008\cite{dresselhaus_group_2008}. For definition in crystallography,
we refer to U. M\"{u}ller 2013\cite{muller_symmetry_2013}. 

\subsection{Groups and Irreducible Representations}
In this work we consider site symmetry groups $S_{\mathbf{x}}$ whose group elements are 
symmetry operation that leave the crystal structure as well as the point $\mathbf{x}$ 
invariant\cite{muller_symmetry_2013}. 
We define a linear representation of group $G$ as a homomorphism:
\begin{definition}
    A linear \emph{representation} of a group $G$ on a vector space $V$ is a homomorphism:
    \[\rho\colon G \to \text{GL}(V)\]
    where $\text{GL}(V)$ is a group of invertiable linear transformation and 
    homomorphism means a mapping that satisfies $\rho(g_1g_2) = \rho(g_1)\rho(g_2)$.
\end{definition}
We note that the dimension of the vector space $V$ of the representation maybe larger than 3. 
The direct sum of representations are also representation and vice versa, a representation
can be splitted into two representations. 
%Representations are not unique. For example, a change of basis of basis by $\rho \to g^{-1}\rho g$ would 
%lead to an equivalent representation, as can be varified.
However, representations can be uniquely reduced to 
so called \emph{irreducible representations}\cite{serre_linear_1996}.

We first define the notion of \emph{invariant vector space}:
\begin{definition}
    For any vector $v$ in $V$, for any $g\in G$, if we have $\rho(g)v \in V$, then $V$
    is an invariant vector space.
\end{definition} 
The vector space of a representation is itself an invariant vector space, but if there 
exist a subspace $W\subset V$ which is also invariant to $G$, then we call $W$
an \emph{invariant subspace}. 
Furthermore, given $W$ is an invariant subspace of $V$, 
there exist another invariant subspace $U$ of $V$ so that $V$ is given 
by a direct sum: $V = U \oplus W$

By splitting the vector space $V$ into invariant subspaces until it is not possible, we find the 
\emph{irreducible representations} as subspaces of $V$, and the matrix form 
of the reducible representation can be written as the block diagonal form
\begin{equation}
    \rho^{\Gamma}(g) = \left(  
        \begin{matrix}
            \rho^{\Gamma_1} (g) & 0 & \cdots\\
            0 & \rho^{\Gamma_2} (g) & \\
            \vdots &  & \ddots
        \end{matrix}
    \right)
\end{equation}
where $\Gamma$ denote the reducible representation in $V$ and $\Gamma_i\ (i = 1, \dots)$ index 
the irreducible representations.

\subsection{Basis functions}
For each invariant subspaces corresponding to a representation $\Gamma_n$, we write its basis 
functions using the braket notation $|\Gamma_n,j\rangle$. For an irreducible representation with 
dimension $N$, we would have $N$ such basis functions and any vectors in this invariant subspace 
can be written as a linear combination:
\begin{equation}
    |\phi^{\Gamma_n}\rangle = \sum_{i = 1}^N c_i |\Gamma_n,i\rangle
\end{equation} 
Due to the definition of invariant subspace, the matrix representation of $G$ in the invariant 
subspace will have the form:
\begin{equation}
    \rho^{\Gamma_n}(g)_{ij} = \langle \Gamma_n,i | P_g | \Gamma_n,j \rangle
\end{equation}
where $P_g$ is the symmetry operation that transforms basis vectors.

Since Hamiltonian $\mathbf{H}$ of the system has the same symmetry as the system itself, $\mathbf{H}$
commutes with symmetry operation $g \in G$. 
As a consequence, if $|\phi_{\varepsilon_i}\rangle$ an eigenfunction of $H$
with eigenvalue $\varepsilon_i$, $P_g|\phi_{\varepsilon_i}\rangle$ will also be an eigenfunction 
with degenerate eigenvalues. Since $|\phi_{\varepsilon_i}\rangle$ and $P_g|\phi_{\varepsilon_i}\rangle$
together will define some invariant space. Therefore, degenerate eigenfunctions will occupy the same 
irreducible representations. Non-degenerate eigenfunctions would correspond to a one dimensional irreducible 
representation. Interactions are forbidden between 
states belong to different irreducible representations. These are the properties
of irreducible representation that we mainly use\cite{dresselhaus_group_2008}.

\subsection{Characters of irreducible representation}
For each operation in an irreducible representation, we define their character 
as the trace of the operation matrix. We note that for an irreducible representation, 
any change of basis $\rho \to g^{-1}\rho{g}$ will give an equivalent representation. However,
the matrix trace would remain the same. Usually, characters are given for each equivalent 
class of group elements $C_k$, so we denote characters as $\chi^{\Gamma}(C_k)$ for a 
representation $\Gamma$\cite{serre_linear_1996}.

The important property of characters is that the character of different irreducible 
representations are orthogonality to each other. As a result, the reduction of 
reducible representations into irreducible ones can be uniquely determined using the 
characters. We have the following result\cite{dresselhaus_group_2008}:
\begin{gather}
    \chi(C_k) = \sum_{\Gamma_i} a_i \chi^{\Gamma_i}(C_k) \\
    a_i = \frac{1}{h} \sum_{k} N_k \chi^{\Gamma_i}(C_k)^* \chi(C_k)
\end{gather}
where $\chi(C_k)$ is the character of the reducible representation and $\chi^{\Gamma_i}(C_k)$
characters of the irreducible representations. $a_i$ is the decomposition coefficients.
$N_k$ is the number of group operation in the equivalent class and $h$ is the order of the group.

Characters are also used in finding the vector subspace of irreducible representations
as a subspace of some reducible ones. Such projection operator is given by:
\begin{equation}
    P^{\Gamma_i} = \frac{l_i}{h}\sum_R \chi^{\Gamma_i}(R) P_R
\end{equation}
where $R$ denote group operation, $P_R$ the correspond operator and $l_i$ the dimension of the 
representations. This projection operator is used to find symmetry adopted basis functions 
in our application.

\end{document}
