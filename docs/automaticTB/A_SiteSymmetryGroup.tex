\documentclass{article}
\usepackage{amssymb, amsmath, amsthm}
\usepackage[margin=1in]{geometry}
\usepackage{verbatim}
\usepackage{graphicx}
\usepackage{hyperref} % \url \href
\usepackage{docmute}

\usepackage{tikz}
\usetikzlibrary{shapes.geometric, arrows}
\tikzstyle{startstop} = [rectangle, rounded corners, minimum width=2cm,text centered, draw=black]
\tikzstyle{process} = [rectangle, minimum width=2cm, text centered, draw=black]
\tikzstyle{decision} = [diamond, minimum width=2cm, text centered, draw=black]
\tikzstyle{arrow} = [thick,->,>=stealth]

\newtheorem{definition}{Definition}
\newtheorem{theorem}{Theorem}
\DeclareMathOperator{\spn}{Span}

\newtheorem{definition}{Definition}
\newtheorem{theorem}{Theorem}
\DeclareMathOperator{\spn}{Span}

\usepackage[style=chem-acs ,backend=bibtex, sorting=none]{biblatex}
\addbibresource{autoTB.bib}

\begin{document}

\section{Character table for site symmetry groups}

To use symmetry information in the construction of a tight-binding model, we first need to 
find the concrete symmetry group and it's representations. 
In this section, our implementation to find the site symmetry group in crystal, as well as 
its subgroups and characters are described in details.

\subsection{Basic Definition}

For a point $\mathbf{x}$ in a crystal, the subgroup of the space group $G$ that leave $\mathbf{x}$
fixed is called the \emph{site symmetry groups}, which we denote $S_{\mathbf{x}}$. A space group
operation in crystal can be denoted using matrix--column pair $(\mathbf{W},\mathbf{w})$ in a chosen 
basis. Then it is straightforwardly to see that group elements in the site symmetry group only 
contain rotation part. 

Given a rotation matrix $\mathbf{W}$ belonging to a site symmetry group, it can be classified into different
types, each designated by their Hermann--Mauguin (HM) symbols, depending on its trace and determinant. 
The trace $\text{tr}\mathbf{W}$ give information of the rotation angle and determinant indicates whether 
the operation is first or second kind. 
Since trace is invariant under permutation of matrices, it does not depend on any change of basis. 
They are listed in Table \ref{T:operation_properties}. 

\begin{table}[h!]
    \centering
    \caption{Site symmetry types}
    \begin{tabular}{|c|cccccccccc|}
        \hline
        $\det$      & +1 & +1 & +1 & +1 & +1 & -1 & -1 & -1 & -1 & -1 \\ 
        $\text{tr}$ &  3 &  2 &  1 &  0 & -1 & -3 & -2 & -1 &  0 &  1 \\  
        \hline
        type        &  1 &  6 &  4 &  3 &  2 & -1 & -6 & -4 & -3 & -2 = m \\ 
        \hline
    \end{tabular}
    \label{T:operation_properties}
\end{table}

There are in total 32 crystallographic point group types so that the number of distinct site symmetry group types are
also limited to 32. They can be easily identified by counting the types of operation in the group. 
Reference table can be found in Togo and Tanaka\cite{togo_spglib_2018}.

For point groups, its group operation can be uniquely denoted using \emph{Seitz symbols}. The Seitz symbol consists of 
three parts: 1) HM symbol of the operation, 2) its characteristic direction denoted by $[uvw]$ and 3) the sense of rotation $+$ or $-$. 
The sense of rotation indicate whether the rotation is clockwise or anti-clockwise and is used for $4$, $-4$, $3$, $-3$, $6$, $-6$ only.
The characteristic direction (lattice direction) is usually given with respect to different lattice systems. We use hexagonal lattice 
for both hexagonal and rhombohedral lattice. The basis for each lattice can be found in Section 2.2 of \emph{ITA}. 

the number of possible site symmetry is greatly reduced compared to 
that in the moleculars. Possible rotations can be characterized by the trace 
and determinant of the rotation matrix, where the $\det\mathbf{W}$ indicate the 
existance of inversion in the operation and $\text{tr}\mathbf{W}$ gives the angle of rotation. 
Since trace is invariant under permutation, angle of rotation does not depend on the 
basis vector of the symmetry matrix. The characterization is listed by table \ref{T:operation_properties}.

\subsection{Finding the Seitz symbol from Operation Matrix}


The total number of crystallographic point groups in three dimension is 32, and their character tables are 
tabulated either in The point group table of the bilbao crystallographic server \href{https://www.cryst.ehu.es/rep/point.html}{bilbao}. 
In crystallographic server, all the matrix element of the operations are also given explicitly, in the respective basis vector 
and denoted in Seitz symbol. Their characters can also be found from the same webpage.
Seitz symbols are tabulated as well (Seitz symbols for crystallographic symmetry operations)
with the standard basis vectors given in the international table for each crystal system. 
In our implementation, the character table of the 32 point symmetry groups are taken from bilbao server and 
stored. For hexagonal lattice, we use the hexagonal setting.

However, one complication arise when the site symmetry operation does not correspond to the standard basis. For example, 
in a cubic crystal belong to space group $Pm\bar{3}m$ (221), the point $(x,x,x)$ on the body diagonal has site symmetry $3m$
whose rotation matrix is tabulated in hexagonal basis. Therefore, we cannot directly compare the rotation matrix of the 
site symmetry operation with the ones tabulated in the Bilbao server. Instead, we use the following method to generate 
the seitz symbol from a list of rotation matrices, and then use the seitz symbol to assign characters to each symmetry 
operations. The input rotation matrices can correspond to any orientation.

A seitz symbol consist of three parts: type of the symmetry operation, sense of the rotation 
(for rotation $4$, $-4$, $3$, $-3$, $6$, $-6$ only) and the characteristic direction. 
Symmetry type can be found according to table \ref{T:operation_properties}. To find the 
symmetry direction in the standard basis, we first find the symmetry direction in the cartesian coordinate. 

For symmetry $1$ or $\bar{1}$, a point $(0.0, 0.0, 0.0)$ will be assigned. If the symmetry operation is a rotoinversion,
we multiplicity it by $-\mathbf{1}$ to obtain the rotation parts $\mathbf{W}$. 
The symmetry direction is given by finding the eigenvector with eigenvalue $1$. 
\begin{equation}
    \label{E:rotation_axes}
    \mathbf{W}\mathbf{v} = \mathbf{v}
\end{equation}

After computing the directions of all the symmetry operation, we separate them into sets that are related by symmetry operations:
\begin{equation}
    \{\mathbf{v}' \mid \mathbf{W}\mathbf{v}, \mathbf{W} \in \mathbf{R}\}
\end{equation}
All symmetry direction belonging to the same set contain the same symmetry elements, and they correspond to the symmetry directions 
used in Hermann--Mauguin symbol. For example, the result of the classification is listed in table \ref{T:6mmm} for group $6/mmm$
\begin{table}[h]
    \centering
    \caption{Symmetry operation and directions in cartesian coordinate in $6/mmm$}
    \begin{tabular}{|c|l|l|}
        \hline
        Operation & Directions & Symmetry Direction\\
        \hline
        -1,1 & (  0.0  0.0  0.0) & $\mathbf{0}$\\
        6,-3,-6,m,2,3 & (  0.0  0.0  1.0) & $[001]$\\
        m,2 & (-0.5,-0.866,0.0), (1.0,0.0,0.0), (-0.5,0.866,0.0) & $[100],[010],[\bar{1}\bar{1}0]$\\
        m,2 & (0.866,-0.5,0.0), (0.0,1.0,0.0), (-0.866,-0.5,0.0) & $[120],[\bar{2}\bar{1}0],[1\bar{1}0]$\\
        \hline
    \end{tabular}
    \label{T:6mmm}
\end{table}

Then, it is possible to choose the basis for the site symmetry group. The choice of 
the axes are listed in table \ref{T:choice_of_axes}. 
After choosing $\mathbf{a}$ and $\mathbf{b}$, $\mathbf{c}$ is chosen so that the three axes
form a right-handed system: $(\mathbf{a}\times \mathbf{b}) \cdot \mathbf{c} > 0$. 
The direction in cartesian coordinates corresponding to the symmetry direction in the seitz symbol can also be computed and can be 
compared to the symmetry direction found using equation \eqref{E:rotation_axes}. Finally, the 
sense of the rotation can be computed using the method provided by 
\emph{Efficient conversion from rotating matrix to rotation axis and angle by extending Rodrigues’ formula},
by calculating the $\sin\theta$. 

\begin{table}[h]
    \centering
    \caption{Choice of $\mathbf{a}$, $\mathbf{b}$ and $\mathbf{c}$}
    \begin{tabular}{|c|c|}
        \hline
        Groups & Choice\\
        \hline
        \begin{tabular}{c}$2$, $m$, $2/m$, $4$, $-4$, $4/m$\\$3$, $-3$, $6$, $-6$, $6/m$\end{tabular}
            & $z = [001]$ \\ \hline
        $222$, $mm2$, $mmm$ & $x=[100],y=[010],z=[001]$ \\\hline
        $32$, $3m$, $-3m$, $622$, $6mm$, $-6m2$, $6/mmm$ & $\{x, y\} \in \{[100], [010],[\bar{1}\bar{1}0]\}$ \\\hline
        $23$, $m-3$, $432$, $-43m$, $m-3m$ & $\{x,y,z\} = \{[100],[010],[001]\}$ \\
        \hline
    \end{tabular}
    \label{T:choice_of_axes}
\end{table}


\begin{figure}[h]
    \centering
    \begin{tikzpicture}[node distance=2cm]
        \node (start) [startstop] {start};
        \node (pro1) [process, below of=start] {Process 1};
        \draw [arrow] (start) -- (pro1);
        \node (end) [startstop, below of=pro1] {end};
        \draw [arrow] (pro1) -- node[anchor=east] {yes} (end);
    \end{tikzpicture}
    \caption{Workflow}
    \label{F:workflow}
\end{figure}

\subsection{Symmetry reduction}
In our application, we need to find the subgroup of the site symmetry group so that we can separate basis 
uniquely for a representation with more than one dimension. To do so, we need to know which symmetry is 
kept when we move from a group to its subgroup. Finding the subgroups is not a trivial task and we 
manually assign maps from the seitz symbol of the subgroup to the ones in the supergroup. In this way, subgroup
and the correspond characters are readily obtained onces the seitz symbol of the site-symmetry group is known.

\subsection{Subduction}
Subduction enable us to perform symmetry descend that split the irreducible representation into 
different subspaces with different symmetry. 

\end{document}
